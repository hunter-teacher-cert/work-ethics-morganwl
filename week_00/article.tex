\documentclass[12pt]{article}

\usepackage{hyperref}

\usepackage{../../hunter}
\usepackage{../../homework}

\author{Morgan Wajda-Levie}
\course{CSCI 77800: Ethics and Computer Science}
\date{September 8, 2021}
\title{Article Summary}

\begin{document}

\maketitle

The article \emph{Amazon's New Algorithm Will Set Workers Schedules According
to Muscle Use} was written by Edward Ongweso, Jr. and published in Vice on
April 15, 2021.
\mbox{\fontsize{7}{7}\url{https://www.vice.com/en/article/z3xeba/amazons-new-algorithm-will-set-workers-schedules-according-to-muscle-use}}

The article describes a letter sent to shareholders by Amazon founder and CEO
Jeff Bezos, in which Bezos proposes a new solution to reducing the risk of
injury to warehoue workers: an adjustment to the worker scheduling
algorithm which would rotate workers through different positions in order to
evenly distribute repetitive stress across the body. The article notes the
frequency of injuries in Amazon warehouses, many of which are caused by
repetitive motion, and Bezos's planned algorithmic change would be ``central
to the company's strategy going forward.''

I worked in agriculture and construction for much of my youth, which gave me a
great appreciation for work that degrades our bodies for minimal pay. Later, I
entered the film industry, where I discovered work with punishing hours that
took its toll emotionally and mentally. (Though long hours ``chained to a
desk'' have left their own physical marks.) I have been thankful for my union,
which grants me better pay and benefits, and allows me to fight for
restitution when an employer mistreats or shortchanges me.

I am concerned about technologies that sit between workers and supervisors or
consumers, particularly automated decision-making tools.  Amazon's
worker-scheduling algorithms have received scrutiny in the past as one of the
causes for poor working conditions at Amazon facilities---employee performance
and behavior is measured automatically, and employees who do not meet
performance benchmarks are automatically fired.  In an
\href{https://www.theverge.com/2019/4/25/18516004/amazon-warehouse-fulfillment-centers-productivity-firing-terminations}{article
for Verge} employees describe a grueling work environment. I was amazed to see
Amazon's proposed solution to poor safety and overwork to be tuning their
algorithms, not to reduce employee workload, but simply to adjust employee
workload \emph{for even greater efficiency.} This only further serves to embed
human labor into mechanized systems.  Amazon is a notoriously bad actor in
this area, but the problem extends to other companies as well, such as
``sharing economy'' platforms like Uber.

How do we move into a technological future that empowers and uplifts working
people, instead of allowing the few who control major technological platforms
to enrich themselves through worker exploitation?

\end{document}
