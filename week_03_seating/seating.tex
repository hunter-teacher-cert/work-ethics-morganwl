\documentclass{article}

\usepackage{../../hunter}
\usepackage{../../homework}

\author{Morgan Wajda-Levie}
\course{CSCI 77800: Ethics and Computer Science}
\date{October 6, 2021}
\title{Seating}

\begin{document}

\maketitle

The simplest change to the algorithm is to make \texttt{seat\_economy}
aware of the buyers' original block size, and attempt to seat economy
passengers by block first, if possible. If no suitable block of free
seats is available, passengers can be seated wherever possible. An
implementation of this algorithm is included.

A slight improvement over this algorithm could be made by sorting
economy passengers by block size before attempting to seat them, so that
largest groups are seated first. One unintended consequence of this
would be that a family of 5 might buy their tickets last, get the only
row of 5 adjacent seats, and keep two families, one of 3 and one of 2,
from sitting together, even if they had purchased their tickets first.

An even better seating algorithm would require a change to the seating
policy --- instead of allowing all Economy Plus passengers the
opportunity to choose their seats before any economy passengers are
seated, a block of Economy passengers would be seated as soon as only
one available block for a group of their size remains. I leave the
implementation of this algorithm as an exercise for myself at a later
date.

\end{document}
