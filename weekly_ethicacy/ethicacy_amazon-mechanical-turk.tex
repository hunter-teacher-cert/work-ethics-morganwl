\documentclass{article}

\usepackage{hunter}
\usepackage{homework}


\author{Jiyoon Kim and Morgan Wajda-Levie}
\course{CSCI 77800: Ethics and Computer Science}
\date{October 27, 2021}
\title{Weekly Ethicacy: Amazon Mechanical Turk}

\begin{document}

\maketitle

We chose to look at Amazon Mechanical Turk, a crowdsourcing or
microworking platform that allows employers (known as requesters) to
order a large volume of small human intelligence tasks, that are
completed by workers earning sometimes as little \$0.01 per task, and
averaging significantly less than the US national minimum wage.
According to a 2016 Pew Research Center survey, 25\% of respondents used
MTurk as a primary source of income. A 2011 paper published with the
Association for Computational Linguistics found that 80\% of tasks were
performed by the most active 20\%, who all spent at least 15 hours a
week on the site.

Amazon Mechanical Turk is ubiquitous in machine learning, both in
research and industry, as well as social, linguistic and psychological
research. (Morgan has seen AMT suggested as a tool in every one of their
Machine Learning textbooks.) Like many other platforms in the gig
economy, Amazon Mechanical Turk provides an impersonal, technological
buffer between employers and thousands of nameless workers, allowing
those employers to pay dramatically substandard wages while enjoying
little awareness of the pay and conditions of their workers. The
technological company in the middle collects a 20\% fee for providing
the service.

\subsection*{Sources}
\begin{itemize}
    \item Semuels, Alana. New York Times. I Found Work on an Amazon
        Website. I Made 97 Cents an Hour.
        \\https://www.nytimes.com/interactive/2019/11/15/nyregion/amazon-mechanical-turk.html
    \item Newman, Andy. The Atlantic. The Internet Is Enabling a New
        Kind of Poorly Paid Hell.
        \\ https://www.theatlantic.com/business/archive/2018/01/amazon-mechanical-turk/551192/
    \item Pew Research Center. Research in the Crowdsourcing Age, a Case
        Study.
        \\ https://www.pewresearch.org/internet/2016/07/11/research-in-the-crowdsourcing-age-a-case-study/
    \item Hara, Kotaro et al. ACM. A Data-Driven Analysis of Workers’
        Earnings on Amazon Mechanical Turk.
        \\https://arxiv.org/pdf/1712.05796.pdf
    \item Fort, Karën et al. ACL. Amazon Mechanical Turk: Gold Mine or
        Coal Mine?.
        \\https://direct.mit.edu/coli/article/37/2/413/2101/Amazon-Mechanical-Turk-Gold-Mine-or-Coal-Mine
    \item Turkopticon. https://turkopticon.net/
\end{itemize}


\end{document}
